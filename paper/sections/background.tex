Recurrent neural networks (RNNs) are Turing-Complete models that can perform
complex transformations on the data over an extended period of time
\cite{siegelmann1995computational}. Since RNNs are Turing-Complete, it is
evident that these models in theory can be used to solve any problems a
Turing machine can solve. However, a given problem may require a very specific
and complex setup of an RNN. Therefore we do not take the time to set up
RNNs manually in a way that it solves a specific problem. Rather, we design
rich architectures with the means of learning how to solve a problem in
an efficient manner. RNNs and models alike are trained using gradient descent
so that it solves a given problem with greater accuracy than before.

\section{Neural Turing Machines}
The Neural Turing Machine (NTM) is an RNN with a rich design that enables it to
learn how to solve problems better than basic RNNs. The NTM is an RNN with the
means of addressing a differentiable memory unit external to that of the model
itself \cite{DBLP:journals/corr/GravesWD14}.
Because of this differentiable memory unit, the model can learn what is most
important to store in memory. An efficient model would learn what information
is most useful and store that in memory to be read at a later timestep with the
assumption that the read results in a better solution to the task.
This creates a model that can more strongly refer to important imformation
that was present much earlier than a given timestep.

The NTM can be described as four main components: controller, read heads,
write heads, memory. The data flow of these four components is described
in Figure ~\ref{fig:NTMArchitecture}. Even though these components work in a
non-traditional manner, the model as a whole can be interacted with and
trained the same way you would any other RNN. The model simply receives
some external input at each timestep and will produce an external output
at each timestep. The components of the NTM are comparable to that of a
Turing machine, such that it contains read and write heads that interact
with a memory unit \cite{DBLP:journals/corr/GravesWD14}. Unlike a Turing
machine however, the NTM is end to end differentiable
\cite{DBLP:journals/corr/GravesWD14}. This feature gives the model a lot of
power in the field of Deep Learning research.

\begin{figure}[!htb]
    \centering
    \includegraphics[width=0.9\linewidth]{resources/NTM.png}
    \caption{
        The components inside the dotted line represent everything the NTM
        is. The arrows represent the control flow of the data starting
        as \textit{External Input} fed into the \textit{Controller}. The
        \textit{Controller} can then activate either \textit{Read Heads} or
        \textit{Write Heads}. The \textit{Read Heads} do not receive any
        information about the \textit{External Input}, but is rather told what
        to read in \textit{Memory} by the \textit{Controller}. The
        \textit{Controller} can send information from the
        \textit{External Input} to the \textit{Write Heads} which then
        write that information into memory. The \textit{Controller} receives
        data back from the \textit{Read Heads} and then produces an
        \textit{External Output} \cite{DBLP:journals/corr/GravesWD14}.
    }
    \label{fig:NTMArchitecture}
\end{figure}

The memory unit is a matrix where the rows represent a memory location and the
number of columns correspond to how large each memory location is
\cite{DBLP:journals/corr/GravesWD14}. The read head emits a weighting vector
the length of the number of memory locations at each timestep
\cite{DBLP:journals/corr/GravesWD14}. A read operation on the memory matrix
is then a convex combination between this weighting and the contents of the
memory matrix \cite{DBLP:journals/corr/GravesWD14}. Using a weighting vector
here empowers the read head to narrowly focus on a single memory location's
information or to use a wider focus by using a little bit of information
from many different memory locations. The differentiability of the model allows
the read heads to learn the best weighting vectors to emit at a given state for
some timestep.

Writing to memory in the NTM is done with an erase operation followed by an
addition operation \cite{DBLP:journals/corr/GravesWD14}. The write head
emits three vectors at each timestep: \textit{weighting vector, erase vector,
add vector} \cite{DBLP:journals/corr/GravesWD14}. Assuming a weighting of $1$
for a memory location, a value of $1$ in the \textit{erase vector} at location
$j$ would mean to set the memory to $0$ in column $j$ within that memory
location \cite{DBLP:journals/corr/GravesWD14}. However, an
\textit{erase vector} with a value of $1$ can possibly not erase any
information in a memory location if the weighting for that memory location is
$0$ \cite{DBLP:journals/corr/GravesWD14}. So the \textit{erase vector}
represents the degree to which information in columns for weighted memory
locations shall be erased. As one would suspect, the \textit{add vector} works
quite oppositely from that of the \textit{erase vector}. Again assuming a
weighting of $1$ for a memory location, a value of $1$ in the
\textit{add vector} at location $j$ would mean to add $1$ to whatever value
is already stored in that memory location in column $j$
\cite{DBLP:journals/corr/GravesWD14}. Similarly to that of the
\textit{erase vector}, the weighting can overrule the \textit{add vector} by
having a value of $0$ for some location \cite{DBLP:journals/corr/GravesWD14}.
In other words, a value of $0$ in the \textit{weighting vector} means it does
not matter what the \textit{add vector} is because nothing will be added to
that memory location.

The weighting vectors emitted by both read and write heads are done so
through two different addressing mechanisms: \textit{content-based addressing},
\textit{location-based addressing} \cite{DBLP:journals/corr/GravesWD14}.
Content-based addressing compares a key vector with all of the memory locations
\cite{DBLP:journals/corr/GravesWD14}. This produces a weighting that has
greater values in locations that are most similar to that key vector and
lesser values in locations that are least similar to that key vector
\cite{DBLP:journals/corr/GravesWD14}. This addressing mechanism is useful for
easily identifying data in the memory matrix simply by producing an
approximation of that data. The location-based addressing mechanism is better
suited for iterating through sequence data or even jumping to a specific
memory location that may have a range of unknown data stored there. If the
previous weighting for a head was focused on location $j$, the location-based
addressing mechanism can easily produce a new weighting that focuses on
location $j-1$, $j$, or $j+1$. The choice between using these two addressing
mechanisms is controlled through a gating parameter emitted by the controller.
If this gating parameter is equal to $0$, then it will only use location-based
addressing. If the gating parameter is equal to $1$, then it will only use
content-based addressing. If the gating parameter is somewhere between
$0$ and $1$, it will use a mixture of both addressing mechanisms.

\section{Differentiable Neural Computers}
The Differentiable Neural Computer (DNC) is similar to that of the NTM, but
enhanced in it's ability to address the augmented external memory. These
enhancements make the DNC more capable than the NTM at solving complex,
structured tasks. Similarly to the NTM, the DNC is differentiable end-to-end
except one operation in which the researchers claim do not make a significant
impact on the model as a whole \cite{graves2016hybrid}.

A NTM is allowed any number of read and write heads, but the DNC limits
the number of write heads to be only one \cite{graves2016hybrid}. The NTM
was only able to iterate through consecutive written locations until a
content-based addressing write occured \cite{graves2016hybrid}. The DNC,
however, can continue iterating through consecutive written locations because
it has, what the researchers call, a temporal linkage matrix
\cite{graves2016hybrid}. The temporal linkage matrix, $L$, is $N \times N$ with
weighted values between zero and one \cite{graves2016hybrid}. A value closer
to one at position $(i, j)$ means memory location $i$ was written to directly
after memory location $j$ was written to \cite{graves2016hybrid}. Otherwise,
the value should be near zero \cite{graves2016hybrid}. So given a weighting
vector, $\bf{w}$, the vector produced by $L\bf{w}$ augments the weighting to
focus on memory locations written to directly \textit{after} those
locations focused on by weighting $\bf{w}$ \cite{graves2016hybrid}.
Conveniently, the vector produced by $L^\top \bf{w}$ augments the weighting to
focus on memory locations written to directly \textit{before} those
locations focused on by weighting $\bf{w}$ \cite{graves2016hybrid}. This
interaction between the temporal linkage matrix and the external memory is one
of the three differentiable attention mechanisms in the DNC
\cite{graves2016hybrid}. This attention mechanism replaces the location-based
addressing method from the NTM. The temporal linkage matrix allows the DNC to
better handle sequential data that is not necessarily written to in sequential
memory locations. This is a more powerful concept than that of the NTM's
location-based addressing mechanism which was limited to sequential
memory locations.

Like the NTM, the DNC uses a content-based addressing mechanism that produces
a weighting vector focusing on memory locations most similar to some
emitted key vector \cite{graves2016hybrid}. Similarity in both cases are
defined as cosine similarity
\cite{DBLP:journals/corr/GravesWD14,graves2016hybrid}. If the DNC controller
emits a key vector with only part of the information it needs, this
content-based addressing mechanism can be used to read the complete
information related to it. In theory this allows the DNC to iterate
through data structures in which memory locations reference other memory
locations through a key-value address mechanism.

Unlike the NTM, the DNC uses a \textit{usage vector} to find the best memory
locations to write to and to identify any important memory locations not to
write to \cite{graves2016hybrid}. The vector's values range between zero and
one, where values closer to one means the memory locations have greater usages
\cite{graves2016hybrid}. The weighting vector given to a write head focuses on
unused locations, where the usage vector has low values
\cite{graves2016hybrid}. When the write head writes to a memory location, the
usage vector is updated at that memory location by increasing the value
\cite{graves2016hybrid}. The usage vector is also capable of decreasing values
in a memory location after it is read \cite{graves2016hybrid}. For example,
consider a task which provides a sequence of numbers and then asks what that
sequence was. After asking for what a sequence was, it will provide a new
sequence of numbers and repeat itself. The DNC can learn to write that
sequence of numbers in the external memory using the write heads. At that
point, the usage vector will have greater values in those locations written
to and lower values in the locations not needed. When the task asks for the
sequence of numbers, the DNC would use the read heads to read those locations
written to. After the reads, the usage vector will update the memory locations
to have low usage. This effectively reallocates the memory locations for
later usage.

The DNC is very comparable to how a human interacts with what we think of as
a modern day computer. Much of the human interaction with computers are done
automatically with a trained DNC. For instance, a human controls more or less
exactly what to read and write in memory. The DNC however, learns what
information is most useful to read and write in memory without being controled
by an external force. When a human deletes data on a computer, they are
essentially reallocating space in their memory for future data to be stored.
In this same manner, the DNC automatically learns when to reallocate memory
based on the task given. So as the name suggests, the DNC can be thought of
as a differentiable computer. This does not mean that a DNC will ever take the
place of a modern day computer because the applications are different. When
we interact with a computer, we expect the memory to always be there regardless
of how often we may use it. Having the computer automatically erase data
because it thinks you do not need it would be very unfavorable for many users.
However, a DNC is a model that learns how to represent complex data
structures to solve difficult tasks that cannot otherwise be solved.

The DNC has been found to be useful in a variety of experimental applications.
The researchers show that the DNC outperforms the previous best models in
a synthetic question answering task using the bAbI dataset
\cite{graves2016hybrid}. The following example shows that the DNC is able to
combine supporting facts together to come up with a solution
\cite{graves2016hybrid}:
\begin{align*}
\textnormal{\bf{Input: }}  & \quad \textnormal{John is in the playground. John
                                               picked up the football.} \\
\textnormal{\bf{Input: }}  & \quad \textnormal{Where is the football?} \\
\textnormal{\bf{Output: }} & \quad \textnormal{playground.}
\end{align*}
An even more complex task is to still be able to combine supporting facts
but also have resilience to distractors in the input \cite{graves2016hybrid}:
\begin{align*}
\textnormal{\bf{Input: }}  & \quad \textnormal{Sheep are afraid of wolves.
                             Gertrude is a sheep.} \\
                           & \quad \textnormal{Mice are afraid of cats.} \\
\textnormal{\bf{Input: }}  & \quad \textnormal{What is Gertrude afraid of?} \\
\textnormal{\bf{Output: }} & \quad \textnormal{wolves.}
\end{align*}
This example requires the DNC to have textual reasoning and be able to ignore
the last sentence that is unecessary to answer the question. This shows that
the differentiable attention mechanisms in the DNC are working and capable
of handling complex tasks such as this.

To test how the DNC handles data structures, the researchers explored
performance on graph problems: \textit{traversal}, \textit{shortest path},
\textit{inference} \cite{graves2016hybrid}. Before the model was asked to solve
one of these problems, it would first receive a graph as input. The model would
learn how to represent this graph in external memory. When the model is asked
to complete a traversal task, it receives a starting node and a path in the
graph \cite{graves2016hybrid}. The model must then output the ending node of
the path, requiring a traversal through the graph \cite{graves2016hybrid}.
The shortest path problem was given by a start and end node
\cite{graves2016hybrid}. The model must output the in-order path from start to
end. The inference task is given by an incomplete triple
(\quotes{from node}, \quotes{to node}, \quotes{relation}), where one of the
three were missing \cite{graves2016hybrid}. For example, if \quotes{to node} is
missing from the input, the model must infer what node has the given
relationship with \quotes{from node}. In the example of a family tree, one
might ask (\quotes{Jen}, \quotes{\_}, \quotes{MaternalGreatUncle}). For which,
if Joe is Jen's Maternal-Great-Uncle, then the model is expected to output
the triple (\quotes{Jen}, \quotes{Joe}, \quotes{MaternalGreatUncle}). This task
requires the model to not only learn how to traverse the graphs, but to learn
the meaning of relationships.

When the three graph problems were tested on the DNC and a baseline LSTM, the
DNC produced much better results while the LSTM failed to even learn the
easiest task of traversal \cite{graves2016hybrid}. When tested on randomly
generated paths of length seven, the DNC performed with an accuracy of 98.8\%
and the LSTM performed with an accuracy of 37\% \cite{graves2016hybrid}. This
shows how the DNC can solve problems otherwise unsolvable because the DNC has
an external memory with read and write access. The DNC performed at an accuracy
of 55.3\% when tested on shortest paths of length four \cite{graves2016hybrid}.
On the last task, inference, the DNC performed with accuracy 81.8\% when tested
on relationships that connect nodes with distance four \cite{graves2016hybrid}.
