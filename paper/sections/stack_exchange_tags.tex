Automatic summarization is a term used to describe the action of processing
some text to output some form of a meaningful summary in a representative
manner. Extraction-based summarization is a type of automatic summarization in
which the goal is to extract key-terms as the summary of some text.

As a novel application for the DNC, I applied it to the extraction-based
summarization task of generating tags from Stack Exchange questions. I obtained
a dataset from the Facebook Recruiting III - Keyword Extraction\footnote{can be
found at https://www.kaggle.com/c/facebook-recruiting-iii-keyword-extraction/}
competition on Kaggle. The training dataset contains 6,034,195 questions and
the testing dataset contains 2,013,337 questions. The correct answers for the
testing dataset are not publicly available, but submissions can still be scored
on the Kaggle competition. Scores reported for this task are mean F-scores as
determined by the Kaggle submission report. For example, $1.0$ means a perfect
score and $0.0$ means nothing correct at all. When the competition was live
four years ago (August 30, 2013), the winner received a score of $0.81350$.

Each question contains three main features: title, question, tags. The training
dataset includes all three features while the testing dataset only contains the
title and question features. Below is an example from the training dataset:
\blockquote{
\textbf{Title:} How to check if an uploaded file is an image without mime type?
\\ \\
\textbf{Question:} {\textless}p{\textgreater}I'd like to check if an uploaded
file is an image file (e.g png, jpg, jpeg, gif, bmp) or another file. The
problem is that I'm using Uploadify to upload the files, which changes the mime
type and gives a 'text/octal' or something as the mime type, no matter which
file type you upload.%
{\textless}/p{\textgreater}{\textbackslash}n{\textbackslash}n%
{\textless}p{\textgreater}Is there a way to check if the uploaded file is an
image apart from checking the file extension using PHP?%
{\textless}/p{\textgreater}{\textbackslash}n
\\ \\
\textbf{Tags:} php image-processing file-upload upload mime-types}
The data comes directly from user-posted questions with only anonymization
changes. The length of the features can all be different for each question. The
goal for the testing dataset is to receive the title and question features as
input and output the tags feature.
